% This is samplepaper.tex, a sample chapter demonstrating the
% LLNCS macro package for Springer Computer Science proceedings;
% Version 2.20 of 2017/10/04
%
\documentclass[runningheads]{llncs}
%
\usepackage{graphicx}
% Used for displaying a sample figure. If possible, figure files should
% be included in EPS format.
%
% If you use the hyperref package, please uncomment the following line
% to display URLs in blue roman font according to Springer's eBook style:
% \renewcommand\UrlFont{\color{blue}\rmfamily}

\begin{document}
%
\title{Bachelor Dissertation}
%
%\titlerunning{Abbreviated paper title}
% If the paper title is too long for the running head, you can set
% an abbreviated paper title here
%
\author{Kasper Engelen\and
		Jonathan Meyer\and
		Dawid Miroyan\and
		Igor Schittekat}
%
%\authorrunning{F. Author et al.}
% First names are abbreviated in the running head.
% If there are more than two authors, 'et al.' is used.
%
\institute{University of Antwerp}
%
\maketitle              % typeset the header of the contribution
%
\begin{abstract}
This document reports our findings regarding the final dissertation. The sections and subsections correspond to the assignments given to us. In this project we worked with a simulator called Stride, developed at the University of Antwerp. We explore various concepts within computational epidemiology through the use of this program. 

\keywords{Computational Epidemiology  \and Dissertation}
\end{abstract}
%
%
%
\section{Simulation}
\subsection{Stochastic Variation}
The first topic we consider is stochastic variation. Since the simulation uses a pseudo-random number generator, it's useful to inspect the influence of this stochasticity on the results of the simulation. Using the Stan tool which is provided with Stride, we collected data on 100 simulations with an identical configuration file. The only difference between executions is the RNG seed. When plotting the results, it becomes apparent that the amount of new cases per day follows a normal distribution. For the data we collected, we calculated a  mean of 33.33 new cases per day, and a variance of 1196.99. Looking at the plot for the cumulative cases per day, two 'categories' can be distinguished:  'outbreak' scenarios and 'extinction' scenarios. These cases are quite evenly spread, which explains the high variance for new cases per day. In the former, the curve has a sigmoid shape, indicating that the disease succesfully spread among the population. The latter scenario corresponds to the curves that are almost constant and are bounded by a value well under the population size. In these cases, the disease did not manage to spread, resulting in only a few infected people at the end of the simulation. This is likely explained by the initial infectation: if the first person to be infected is sufficiently isolated (either socially or by being surrounded by people who are immune), the disease doesn't have a chance to spread.

\subsection{Extinction Threshold}
As discussed previously, it might be the case that only very few people become infected over the course of a simulation. This is referred to as extinction. There is a clear distinction between outbreaks and extinctions, so in this subsection we attempt to find an \emph{extinction threshold}.

\paragraph{}
Figure \ref{ETHist} gives the frequencies of the amount of infected people at the end of the simulations. The distinction between outbreaks and extinctions is quite clear from this histogram. There is one peak on the lower end of the x-axis, and there is a cluster on the higher-end.

\subsection{Immunity Level}
Placeholder

\subsection{Estimating $R_{0}$}
Placeholder

\begin{table}
\caption{Table captions should be placed above the
tables.}\label{tab1}
\begin{tabular}{|l|l|l|}
\hline
Heading level &  Example & Font size and style\\
\hline
Title (centered) &  {\Large\bfseries Lecture Notes} & 14 point, bold\\
1st-level heading &  {\large\bfseries 1 Introduction} & 12 point, bold\\
2nd-level heading & {\bfseries 2.1 Printing Area} & 10 point, bold\\
3rd-level heading & {\bfseries Run-in Heading in Bold.} Text follows & 10 point, bold\\
4th-level heading & {\itshape Lowest Level Heading.} Text follows & 10 point, italic\\
\hline
\end{tabular}
\end{table}


\noindent Displayed equations are centered and set on a separate
line.
\begin{equation}
x + y = z
\end{equation}

\begin{theorem}
This is a sample theorem. The run-in heading is set in bold, while
the following text appears in italics. Definitions, lemmas,
propositions, and corollaries are styled the same way.
\end{theorem}
%
% the environments 'definition', 'lemma', 'proposition', 'corollary',
% 'remark', and 'example' are defined in the LLNCS documentclass as well.
%
\begin{proof}
Proofs, examples, and remarks have the initial word in italics,
while the following text appears in normal font.
\end{proof}
For citations of references, we prefer the use of square brackets
and consecutive numbers. Citations using labels or the author/year
convention are also acceptable. The following bibliography provides
a sample reference list with entries for journal
articles~\cite{ref_article1}, an LNCS chapter~\cite{ref_lncs1}, a
book~\cite{ref_book1}, proceedings without editors~\cite{ref_proc1},
and a homepage~\cite{ref_url1}. Multiple citations are grouped
\cite{ref_article1,ref_lncs1,ref_book1},
\cite{ref_article1,ref_book1,ref_proc1,ref_url1}.
%
% ---- Bibliography ----
%
% BibTeX users should specify bibliography style 'splncs04'.
% References will then be sorted and formatted in the correct style.
%
% \bibliographystyle{splncs04}
% \bibliography{mybibliography}
%
\begin{thebibliography}{8}
\end{thebibliography}
\end{document}
