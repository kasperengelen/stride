\documentclass{article}
\usepackage{graphicx}
\usepackage{float}
\usepackage{subcaption}
\captionsetup{compatibility=false}

\begin{document}

\section{Performance profiling}
Using GProf, we will look at the performance in different scenarios. By varying parameters, we try to see which parts of the code they have an influence on, and which parts take up the most time. 
\\
For the first 4 parameters
\begin{itemize}
	\item amount of days
	\item population size
	\item immunity rate
	\item seeding rate
\end{itemize}
the actual sorting and analyzing of the population takes up most of the time.

\subsection{Number of days}
By increasing the number of days to be simulated, the total execution time gets longer as well. This should be expected as more days means more times simulating what goes on in a day.
\begin{table}[!h]
	\centering
	\begin{tabular}{|c|c|}
		\hline
		Number of days & Time \\\hline
		50  & 00:00:04 \\\hline
    	100 & 00:00:09 \\\hline
    	150 & 00:00:14 \\\hline
    	200	& 00:00:18 \\\hline
    	500 & 00:00:38 \\
    	\hline
	\end{tabular}
	\caption{Number of days}
\end{table}

\subsection{Population size}
Generating a new population depends on the given size, which was expected. The generation however is very fast.
\begin{table}[!h]
	\centering
	\begin{tabular}{|c|c|}
		\hline
		Population size & Time \\\hline
	    10000  & 00:00:00:421 \\\hline
	    50000  & 00:00:00:816 \\\hline
	    100000 & 00:00:01:332 \\\hline
 	    200000 & 00:00:02:249 \\\hline
 	    500000 & 00:00:04:926 \\
    	\hline
	\end{tabular}
	\caption{Population size}
\end{table}

\subsection{Immunity rate}
Varying the immunity rate does not seem to affect the total execution time.
\begin{table}[!h]
	\centering
	\begin{tabular}{|c|c|}
		\hline
		Immunity rate & Time \\\hline
    	0.2  & 00:00:05:817 \\\hline
    	0.4  & 00:00:05:750 \\\hline
    	0.6  & 00:00:05:643 \\\hline
    	0.8  & 00:00:05:580 \\\hline
    	0.99 & 00:00:05:648 \\
    	\hline
	\end{tabular}
	\caption{Immunity rate}
\end{table}

\subsection{Seeding rate}
A bigger seeding rate slightly increases the time of the execution. This happens as more people have to be initially infected.
\begin{table}[!h]
	\centering
	\begin{tabular}{|c|c|}
		\hline
		Seeding rate & Time \\\hline
    	0.02 	& 00:00:06:599 \\\hline
    	0.002 	& 00:00:05:589 \\\hline
    	0.0002 	& 00:00:05:134 \\\hline
  		0.00002 & 00:00:04:865 \\
    	\hline
	\end{tabular}
	\caption{seeding rate}
\end{table}

\subsection{Contact log mode}
The mode of the contact log has a very large impact on the execution time. Logging every contact between people takes a long time.\\
At day 50 in the simulation, only 20000 people out of 600000 where infected. When logging the susceptible people, you actually log almost 580000 people at each day which is very close to logging all people. This is very fast when the mode is set to Transmissions as you would only log once for each newly infected person.\\

\begin{table}[!h]
	\centering
	\begin{tabular}{|c|c|}
		\hline
		Contact log mode & Time \\\hline
   		All 			& 00:26:48:251 \\\hline
   		Transmissions 	& 00:00:06:730 \\\hline
   		Susceptibles 	& 00:27:06:444 \\\hline
   		None 			& 00:00:05:824 \\
    	\hline
	\end{tabular}
	\caption{Contact log mode}
\end{table}

\end{document}
